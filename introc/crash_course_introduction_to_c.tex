\section{Crash course introduction to C}

The only way to start learning C is by starting with the \emph{hello world} program. As per the original example that Kernighan and Ritchie proposed way back when, this program hasn't changed that much.

\begin{lstlisting}[language=C]
#include <stdio.h>
int main(void) {
    printf("Hello World\n");
    return 0;
}
\end{lstlisting}

\begin{enumerate}
	\item The \keyword{\#include} directive takes the file \keyword{stdio.h} (which stands for \textbf{st}an\textbf{d}ard \textbf{i}nput and \textbf{o}utput) located somewhere in your operating system, copies the text, and substitutes it where the \keyword{\#include} was.
	\item The \keyword{int main(void)} is a function declaration. The first word \keyword{int} tells the compiler what the return type of the function is. The part before the parenthesis (\keyword{main}) is the function name. In C, no two functions can have the same name in a single compiled program, although shared libraries are a different touchy subject. Then, what comes after is the parameter list. When we provide the parameter list for regular functions \keyword{(void)} that means that the compiler should produce an error if the function is called with a non-zero number of arguments. For regular functions having a declaration like \keyword{void func()} means that you are allowed to call the function in a manner similar to \keyword{func(1, 2, 3)}, because there is no delimiter. \keyword{main} is a special function. There are many ways of declaring \keyword{main} but the ones that you will likely be familiar with are \keyword{int main(void)}, \keyword{int main()}, and \keyword{int main(int argc, char *argv[])}.
	\item \keyword{printf("Hello World\n");} is what we call a function call. \keyword{printf} is defined as a part of \keyword{stdio.h}. The function has been compiled and lives somewhere else on our machine (the location of the C standard library). All we need to do is include the header and call the function with the appropriate parameters (a string literal \keyword{"Hello World\n"}). If you don't include a newline, the buffer will not be flushed (i.e. the write will not complete immediately). It is by convention that buffered IO is not flushed until a newline.
	\item \keyword{return 0;}. \keyword{main} has to return an integer. By convention, \keyword{return 0} means success and anything else means failure. Here are some exit codes / statuses with special meaning: \link{http://tldp.org/LDP/abs/html/exitcodes.html}. Note that all of CS 241's assignments require an exit code to be 0 to indicate successful execution, and normal exit. 
\end{enumerate}

\begin{lstlisting}[language=bash]
$ gcc main.c -o main
$ ./main
Hello World
$
\end{lstlisting}

\begin{enumerate}
	\item \keyword{gcc} is short for the GNU Compiler Collection which has a host of compilers ready for use. The compiler infers from the extension that you are trying to compile a .c file.
	\item \keyword{./main} tells your shell to execute the program in the current directory called main. The program then prints out "hello world".
\end{enumerate}



\section{History of C}

C was developed by Dennis Ritchie and Ken Thompson at Bell Labs back in 1973 \cite{Ritchie:1993:DCL:155360.155580}.
Back then, we had gems of programming languages like Fortran, ALGOL, and LISP.
The goal of C was two fold.
One, to target the most popular computers at the time like the PDP-7.
Two, try and remove some of the lower level constructs like managing registers, programming assembly for jumps and instead create a language that had the power to express programs procedurally (as opposed to mathematically like lisp) with more readable code all while still having the ability to interface with the operating system.
It sounded like a tough feat.
At first, it was only used internally at Bell Labs along with the UNIX operating system.

The first "real" standardization is with Brian Kernighan and Dennis Ritchie's book \cite{kernighan1988c}. It is still widely regarded today as the only \gls{portable} set of C instructions. The K\&R book is known as the de-facto standard for learning C.  There were different standards of C from ANSI to ISO after the Unix guides. The one that we will be mainly focusing on is the \gls{POSIX} C library. Now to get the elephant out of the room, the Linux kernel is not entirely POSIX compliant. Mostly, it is because they didn't want to pay the fee for compliance but also it doesn't want to be completely compliant with a bunch of different standards because then it has to ensue increasing development costs to maintain compliance.

Fast forward however many years, and we are at the current C standard put forth by ISO: C11.
Not all the code that we us in this class will be in this format.
We will aim to using C99 as the standard that most computers recognize.
We will talk about some off-hand features like \keyword{getline} because they are so widely used with the GNU-C library.
We'll begin by providing a decently comprehensive overview of the language with pairing facilities.

\section{Features}

\begin{itemize}
	\item Fast. There is very little separating you and the system.
	\item Simple.
    C and its standard library pose a simple set of portable functions.
	\item Memory Management.
    C let's you manage your memory.
    This can also bite you if you have memory errors.
	\item It's Everywhere.
    Pretty much every computer that is not embedded has some way of interfacing with C.
    The standard library is also everywhere.
    C has stood the test of time as a popular language, and it doesn't look like it is going anywhere.
\end{itemize}



\section{The C and Linux}

So up until this point we've covered language fundamentals with C.
What starts diverging from all forms of C is the functions we use and how we interact with the operating system.
We'll now be focusing our attention to C and the POSIX variety of functions available to us.
We will talk about portable functions, for example \texttt{fwrite} \texttt{printf} etc etc, but we will be evaluating the internals and scrutinizing them under the POSIX models and more specifically LINUX. There are a number of things to that philosophy that makes the rest of this easier to know, so we'll put those things here.

\subsection{Everything is a File}

One POSIX mantra is that everything is a File.
Although that has become recently outdated, and moreover wrong, it is the convention we still use today.
What POSIX means is everything is a file is that everything is a file descriptor or an integer. For example, here is a file object, a network socket, and a kernel object

\begin{lstlisting}[language=C]
  int file_fd = open(...);
  int network_fd = socket(...);
  int kernel_fd = epoll_create1(...);
\end{lstlisting}

And operations on those objects are done through system calls.
One last thing to note before we move on is that the file descriptors are merely \textit{pointers}.
Imagine that each of the file descriptors in the example actually refer to an entry in a table of objects that the operating system picks and chooses from.
Objects can be allocated and deallocated, closed and opened, etc.
The program interacts with these objects by using the API or system calls specified

\subsection{System Calls}

Before we dive into common C functions, we need to know what a system call is.
If you are a student and completed HW0, feel free to gloss over this section.

A system call is an operation that the kernel carries out instead of the program.
The operating system prepares a system call then the kernel executes the system call to the best of its ability in kernel space.
In the previous example we opened up a file descriptor object.
We can now also write some bytes to the file descriptor object that represents a file and the operating system will do its best to get the bytes written to the disk.

\begin{lstlisting}[language=C]
write(file_fd, "Hello!", 6);
\end{lstlisting}

When we say the kernel tries its best, the operation could fail for a number of reasons. The file is no longer valid, the hard drive failed, the system was interrupted etc etc.
The way that you communicate with the outside system is with system calls though.
The other thing to note is that system calls are expensive.
Their cost in terms of time and CPU cycles has recently been decreased, but try to use them as sparingly as possible.

\subsection{C Calls}

Many C calls that we will discuss in the next sections will call some of the calls above in their Linux implementation.
Their Windows implementation may be entirely different.
But we will be looking at one operating system in general.



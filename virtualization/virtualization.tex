\chapter{Virtualization}

The heart of virtualization is a simple concept.
Instead of having the installed operating system run on physical hardware, we can have the OS run on virtual hardware.
Virtualization typically refers to creating a completely virtualized environment.
This means that we emulate everything that the operating system needs.
There are have been new types of virtualization called containers that only virtualize specific parts of their application.

\section{Traditional Applications}

First, we'll talk about traditional virtualization.
Usually to virtualize an operating system, you install a program that handles the virtualization.
Some common ones include VMWare, Virtual Box, and Qemu.
Some of these only support what is known as ``bare metal'' installation.
This means that the software is installed directly on the hardware, and one can run virtual machines on top of them.
The other type installs on top of an existing operating system.
There are benefits and tradeoffs to bare metal installation versus on top of operating systems that we won't cover fully, but some examples are

\begin{enumerate}
\item Bare metal installations are often faster and better at managing resources because they don't have to fight with an operating system
\item Bare metal solutions can be complex as it needs to support many different hardware vendors
\item Non Bare metal systems allow users to run other programs in the operating system next to the virtual machine.
\end{enumerate}

After the environment is set up, one can set up one or more virtual machines.
Traditional virtual machines install the entire operating system in this virtualized environment, so you'll need an install disk.
After installation, one can configure virtual peripherals like network adapters, hard drives, and even CPUs.
The added benefit is that these systems are secure.
Applications that are spun up in this virtual environment -- if the virtualization environment is programmed correctly -- cannot alter the outside system in malicious ways unless the user has allowed it.

\section{Hypervisors}



\section{Topics}

\section{Review}

\bibliographystyle{plainnat}
\bibliography{virtualization/virtualization}

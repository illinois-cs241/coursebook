\section{The Linux Kernel}

Throughout the course of CS 241, you become familiar with system calls - the userspace interface to interacting with the
kernel. How does this kernel actually work? What is a kernel? In this section, we will explore these questions in more
detail and shed some light on various black boxes that you have encountered in this course. We will mostly be focusing
on the linux kernel in this chapter, so please assume that all examples pertain to the linux kernel unless otherwise
specified.

%Topics to be covered:
%* Microkernel vs monolithic kernel
%* Overview of linux kernel architechture/brief history of development.

\subsection{What kinds of kernels are there?}

As it stands, most of you are probably familar with the linux kernel, at least in terms of interacting with it via
system calls. Some of you may also have explored the windows kernel (which we won't talk about too much in this chapter)
or \keyword{Darwin}, the UNIX-like kernel for macOS (a derivative of BSD). Those of you who might have done a bit more
digging might have also encountered projects such a \keyword{GNU HURD} or \keyword{zircon}.

Kernels can generally be classified into one of two categories, a monolithic kernel or a micro-kernel. A monolithic
kernel is essentially a kernel and all of it's associated services as a single program. A micro-kernel on the other hand
is designed to have a \textit{main} component which provides the bare-minimum functionallity that a kernel needs. This
involves setting up important device drivers, the root filesystem, paging or other functionality that is imperitive for
other higher-level features to be implemented. The higher-level features (such as a networking stack, other filesystems,
and non-critical device drivers) are then implemented as separate programs that can interact with the kernel by some
form of IPC, typically RPC. As a result of this design, micro-kernels have traditionally been slower than monolithic
kernels due to the IPC overhead.

We will devote our discution from here onwards to focusing on monolithic kernels and unless specified otherwise,
\textbf{specifically} the linux kernel.

\subsection{System Calls Demystified}

By now, you probably know that system calls are an instruction that can be run by a program operating in userspace that
\textit{traps} to the kernel (by use of a signal) to perform behavior that the user is not allowed to do directly. This
includes actions such as writing data to disk, interacting directly with hardware in general or operations related to
gaining or relinquishing privliages (e.g.  becoming the root user and gaining all capabilities).

In order to fulfill a user's request, the kernel will rely on \keyword{kernel calls}. Kernel calls are essentially the
"public" functions of the kernel - functions implemented by other developers for use in other parts of the kernel. Here
is a snippet for a kernel call's man page:

\begin{lstlisting}
Name

kmalloc — allocate memory
Synopsis
void * kmalloc (	size_t size,
 	gfp_t flags);

Arguments

size_t size

    how many bytes of memory are required.
gfp_t flags

    the type of memory to allocate.

Description

kmalloc is the normal method of allocating memory for objects smaller than page size in the kernel.

The flags argument may be one of:

GFP_USER - Allocate memory on behalf of user. May sleep.

GFP_KERNEL - Allocate normal kernel ram. May sleep.

GFP_ATOMIC - Allocation will not sleep. May use emergency pools. For example, use this inside interrupt handlers.
\end{lstlisting}

You'll note that some flags are marked as potentially causing sleeps. This tells us whether or not we can use those
flags in special scenarios, like interrupt contexts, where speed is of the essence, and operations that may block or
wait for another process may never complete.

%* How to add a new system call?
%* What are processes and threads according to the kernel?
%* How is a scheduler actually implemented in the kernel?
%* What is a kernel module?
%* Example kernel module.

